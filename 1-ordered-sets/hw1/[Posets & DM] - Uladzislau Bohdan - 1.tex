\documentclass{article}

\usepackage{geometry}
 \geometry{
 a4paper,
 total={170mm,257mm},
 left=20mm,
 top=20mm,
 }

\usepackage[utf8]{inputenc}

\usepackage{amsfonts}
\usepackage{amsmath}

\usepackage[none]{hyphenat}

\usepackage{listings}

\setlength{\parindent}{0pt}
\setlength{\parskip}{1em}

\begin{document}

{\Large

Uladzislau Bohdan

MS program ``Data Science''

Ordered Sets in Data Analysis

\vspace{5mm}

Home assignment \#1

}

\vspace{8mm}

\textbf{Task 1}

By defition, $f: U \rightarrow U$ is a surjective function if
$\forall b \in U$, $\exists a \in U$ and $b = f(a)$.

Let's assume the statement is wrong and $f$ is not necessarily injective. That means
$\exists a_1, a_2 \in U$, $a_1 \neq a_2$, such that $f(a_1) = f(a_2) = b_0$.

Then, because $f$ is surjective, $\forall b \in U/\{b_0\}$, $\exists a \in U/\{a_1, a_2\}$ such that $b = f(a)$.

Because cardinality of a set $U/\{a_1, a_2\}$ is less than such of a set $U/\{b_0\}$,
then by definition of a function and by pigeonhole principle we have a contradiction.

Therefore, if function $f: U \rightarrow U$ is surjective, than it is also injective.

\vspace{8mm}

\textbf{Task 2}

In this task I'll be talking about binary relations $R$ as about matrices $5 \times 5$,
where element $(i, j) = 1$ if $a_i R a_j$, otherwise $(i, j) = 0$. Here $(i, j)$ is an element of the matrix.

\textbf{a) Asymmetric and transitive}

The relations are asymmetric if $(i, j) \neq (j, i)$.

The relations are transitive if $(i, j) = 1$ and $(j, k) = 1 \Rightarrow (i, k) = 1$.

Given such definitions, it follows that:
\begin{itemize}
  \item We should only analyze a triangle matrix with no diagonal included.
  Without loss of generality, we work with upper right triangular matrix.
  Lower left triangular matrix is then uniquely defined due to asymmetric relations.
  \item Diagonal is generated independently from the rest of the matrix and diagonal with any values is valid.
  \item Not all triangular matrices are valid due to transitive relations.
\end{itemize}

The code below calculates the number of valid triangular matrices:

\lstinputlisting[language=Python]{2a.py}

The output is $357$.

Each valid triangular matrix can coexist with any valid diagonal.
Total number of valid diagonals is $2^5 = 32$.

This gives us a total number of relations of $357 * 32 = 11424$.

\textbf{b) Antisymmetric and antireflexive}

The relations are antisymmetric if $(i, j) = 1$ and $(j, i) = 1 \Rightarrow i = j$.

The relations are antireflexive if $(i, i) = 0$.

It follows that:
\begin{itemize}
  \item Diagonal elements are all zeros.
  \item Elements upper (WLG) then a diagonal (any) may be any values;
  for elements lower then the diagonal, if $(i, j) = 1$, then $(j, i) = 0$,
  if $(i, j) = 0$, then $(j, i)$ is either $0$ or $1$.
\end{itemize}

The following code does the calculations:

\lstinputlisting[language=Python]{2b.py}

The output is $59049$ - a total number of matrices which satisfy the conditions.

\end{document}
